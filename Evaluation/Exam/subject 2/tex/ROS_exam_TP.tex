\documentclass[10pt,a4paper,english]{exam}
\usepackage[utf8]{inputenc}
\usepackage[T1]{fontenc}
\usepackage[english]{babel}

% Quelques paquets ...
%\usepackage{color} 
\usepackage[usenames,dvipsnames]{xcolor}
\usepackage{amsmath}
\usepackage{amssymb}
\usepackage{graphicx}
\usepackage{enumitem}
\usepackage{tcolorbox}
\usepackage{framed}
\usepackage[framemethod=tikz]{mdframed}

% Mise en page
\usepackage{geometry}
\geometry{hmargin=2.5cm,vmargin=2cm}

% Quelques commandes
\newcounter{mainmemorder}
\newcommand{\save}{\setcounter{mainmemorder}{\value{enumi}}}
\newcommand{\load}{\setcounter{enumi}{\value{mainmemorder}}}
\newcommand{\mytext}[1]{\colorbox{lightgray}{\texttt{#1}}}

% Examen
\printanswers \shadedsolutions
%\noprintanswers
\newcommand{\bareme}[1]{\textbf{\newline\color{red}#1pts} -}

\global\mdfdefinestyle{graybox}{%
  linecolor=black,linewidth=1pt,%
 % leftmargin=1cm,rightmargin=1cm,%
  backgroundcolor=lightgray
}

\mdfdefinestyle{evaluation}{
    frametitlebackgroundcolor=black!15,
    frametitlerule=true,
    roundcorner=10pt,
    middlelinewidth=1pt,
    innermargin=0.5cm,
    outermargin=0.5cm,
    innerleftmargin=0.5cm,
    innerrightmargin=0.5cm,
    innertopmargin=\topskip,
    innerbottommargin=\topskip,
    frametitle={Evaluation}
}
\newcommand\blfootnote[1]{%
  \begingroup
  \renewcommand\thefootnote{}\footnote{#1}%
  \addtocounter{footnote}{-1}%
  \endgroup
}

% C'est parti .......
\begin{document}

% Titre
{\large
%\noindent NOM/Prénom :  \hspace{6cm} Num. étudiant :\\
\begin{center}
	{
		\textbf{Sorbonne Université -- Département de Sciences De l'Ingénieur\\}
		\textbf{ROS and experimental robotics}}\\
	\emph{February 2023, 1h30. \\
		%\vspace{0.4cm}
	}
\end{center}}

You are working in a Robotics company currently developing a new low-cost drone called \texttt{mydrone},
endowed with 4 propellers, one camera, one temperature sensor, and one pressure sensor. The drone is
controlled from a remote which is only equipped with joysticks and leds indicating the current
status of different elements on the drone, like the battery charging level, or warnings about e.g.\
too far distance between the drone and the pilot.

You are in charge of the altitude estimation of the drone, which must be computed from the
data measured onboard during  the flight. Luckily, the altitude relates directly to the air pressure
and current temperature through the relation
%
\begin{equation}
	P = P_0\ e^{-\frac{\mu g h}{R T}},
	\label{eq:altitude}
\end{equation}
%
where $P_0 = 101325$ Pa is the standard pressure at sea level, $\mu = 0.0289644$ kg/mol is the mean
molecular mass of air, $g=9.80665$ m/s$^2$ is the gravity, $R = 8.31432$ J/K.mol is the perfect gas
constant, $P$ the pressure in Pa, T the temperature\footnote{Remember that $T[K] = 273.15 +
		T[^{\circ}C]$} in K, and $h$ the altitude in m.

Your colleagues have already developed a ROS node \mytext{mydrone} running on the drone, and
communicating directly with the hardware on the system, and publishing the sensors values on
dedicated topics. For now, the engineers are able to publish (i) the temperature measured by the
corresponding sensor in the \mytext{/temperature} topic (in $^\circ C$), and the air pressure around
the drone in the \mytext{/pressure} topic (in Pa). \\

In addition, the ROS node \mytext{mydrone} is also in charge of triggering some LEDs on the remote
controller used by the pilot. Your colleagues have so far only implemented the triggering of the
\texttt{ALT WARN} indicator, which is illuminated in green if the drone flights below 1300m of
altitude, or red if ot flights above. This altitude threshold has been determined thanks to the
range of the remote controller: if the drone is too high, the drone might not be able to respond to
the pilot controls anymore. The change in color of the LED is performed by calling the service
\mytext{/altitude\_warning}, which uses as an input the text string "NORMAL" if the drone altitude
is below 1300m, or "WARNING" if it is above.\\

\textbf{Your objective is to write a ROS node called \mytext{exam}, which must publish an estimation
	of the drone altitude on a new \mytext{/altitude} topic (in m), and trigger if needed the
	\texttt{ALT WARN} indicator on the remote by calling the appropriate service.\\}

\begin{mdframed}[style=evaluation]
	At the end of the exam, you have to upload two files to Moodle:
	\begin{itemize}
		\item A PDF report, written with \texttt{openoffice} for example (in french or english), where
		      you answer to the questions below and where you can copy and past the outputs of the
		      different commands you use to answer the questions. For example, for the question ``6.
		      List all the running topics'', you must put in your report :
		      \begin{itemize}
			      \item the exact command line you use to answer the question,
			      \item and the corresponding output you get.
		      \end{itemize}
		      Obviously, you have to comment all your responses. A simple copy and past without a
		      sentence explaining the outputs is \textbf{not} considered as a full response to the
		      questions.
		\item A \texttt{zip} file of your \texttt{exam} package that will be used to access your
		      code, but also to actually launch it.
	\end{itemize}
\end{mdframed}

\section*{Preparation}
\begin{enumerate}
	\item To begin with, download the file \texttt{mydrone.zip} from Moodle. Extract it in the
	      \texttt{src} folder of your catkin workspace. Launch \mytext{catkin\_make} to build your
	      ROS workspace and source you \texttt{.bashrc}.
	\item Go to the \texttt{src} folder of the package \mytext{mybot} you just downloaded, and make
	      executable all the python scripts in there with the shell command \texttt{chown +x
		      python\_file.py}.
	\item Quickly test that all is OK by running the node \mytext{mydrone\_node}. Contact your
	      teacher if it is not the case.
	      \save
\end{enumerate}


%===================================================================================================
\section{Exploration of ROS and of the nodes}

\begin{enumerate}
	\load
	\item Run the node \mytext{mydrone\_node} from the package \mytext{mydrone}. Then list all the running nodes.
	      \begin{solution} \textbf{\textcolor{red}{1 point. La moitié si pas de commentaire.}}
		      \begin{verbatim}
> rosnode list
/mydrone
/rosout	      			
		\end{verbatim}
		      2 nœuds tournent : le nœud attendu et \mytext{rosout}, lancé avec le master via la
		      commande \texttt{roscore}.
	      \end{solution}
	\item List the information about the \mytext{mydrone\_node} node. Explain each line of the output
	      you get.

	      \begin{solution}\textbf{\textcolor{red}{2 points. 1pt si des lignes ne sont pas expliquées.}}
		      \begin{verbatim}
> rosnode info /mybot_usarray
--------------------------------------------------------------------------------
Node [/mydrone]
Publications:
 * /pressure [std_msgs/Float32]
 * /rosout [rosgraph_msgs/Log]
 * /temperature [std_msgs/Float32]

Subscriptions: None

Services:
 * /altitude_warning
 * /mydrone/get_loggers
 * /mydrone/set_logger_level

contacting node http://127.0.0.1:62496/ ...
Pid: 13996
Connections:
 * topic: /rosout
    * to: /rosout
    * direction: outbound (62498 - 127.0.0.1:62501) [11]
    * transport: TCPROS\end{verbatim}
		      Je ne vais pas tout lister (mais les étudiants doivent le faire !). La dernière partie
		      n'est souvent pas comprise :
		      \begin{itemize}
			      \item \verb+contacting node http://127.0.0.1:56353/ ...+: adresse et port où
			            tourne le nœud
			      \item \verb+Pid: 12097+: Process Identification du numéro du processus
			      \item  \verb+* direction: outbound (56355 - 127.0.0.1:56358) [10]+: type de
			            connexion, et vers quelle addresse/port (ici, celui du master)
			      \item \verb+* transport: TCPROS+: protocole utilisé par ROS, ici une connection
			            TCPROS.
		      \end{itemize}
	      \end{solution}
	\item List all the available topics. Explain the role of the \mytext{/rosout} topic.
	      \begin{solution}\textbf{\textcolor{red}{1 point. Aucun si /rosout n'est pas expliqué (ou explication fausse).}}
		      \begin{verbatim}
> rostopic list
/pressure
/rosout
/rosout_agg
/temperature
			   \end{verbatim}
		      \texttt{/rosout} is le topic sur lequel les nœuds peuvent publier des messages de log de manière
		      centralisée.
	      \end{solution}

	\item What kind of message is available on the \mytext{/temperature} topic?
	      \begin{solution}\textbf{\textcolor{red}{1 point. La moitié si pas de commentaire.}}
		      \begin{verbatim}
> rosmsg info std_msgs/Float32
float32 data
		\end{verbatim}
		      Valeur \texttt{data} qui stocke la valeur du message en float.
	      \end{solution}
	\item Display a few messages available on the topic \mytext{/temperature}. At which rate are they
	      published?
	      \begin{solution}\textbf{\textcolor{red}{1 point. 0.5 si commande bonne mais rate faux.}}
		      \begin{verbatim}
> rostopic echo /temperature
data: 2.8841073513031006
---
data: 2.48634672164917
---
data: 2.0330379009246826
---
> rostopic hz /temperature
subscribed to [/temperature]
average rate: 0.777
	min: 1.286s max: 1.286s std dev: 0.00000s window: 2
average rate: 0.777
	min: 1.286s max: 1.288s std dev: 0.00066s window: 3\end{verbatim}
		      Publication à 0.77Hz environ.
	      \end{solution}
	      \save
\end{enumerate}

%===================================================================================================
\section{The exam node}

You have to write a new ROS node \mytext{exam} which must compute and publish the altitude of the
drone.

\begin{enumerate}
	\load
	\item Create a package \mytext{exam} in your \texttt{catkin} workspace, with dependencies to \texttt{rospy}, \texttt{std\_msgs} and
	      \texttt{mydrone} ;
	      \begin{solution}
		      \textbf{\textcolor{red}{1 point. 0 si rien  n'est expliqué.}}\\
		      \texttt{catkin\_create\_pkg exam rospy std\_msgs mydrone}
	      \end{solution}
	\item Create the node \mytext{exam\_node} inside the package \mytext{exam} which must subscribe
	      to the two topics \mytext{/temperature} and \mytext{/pressure}. To begin with, you might only
	      display the temperature and pressure values received on these topics.
	      \begin{solution}
		      \textbf{\textcolor{red}{2 points. 1 point si seulement un topic.}}
		      Voir la proposition de solution \texttt{exam\_node.py}. Ici, il faut en fait que les
		      étudiants aient défini 2 subscriber, et les 2 callback qui vont avec, qui peuvent à ce
		      stade ne faire qu'un print du message reçu.
	      \end{solution}
	      \save
\end{enumerate}

\begin{mdframed}[style=evaluation]
	It is important you comment \textbf{extensively} your code, even for elementary lines of code.
	Your node will be assessed by running it (it must obviously run as expected and provide the
	functionalities listed above), but the quality of the code and its comments will also be taken
	into account.
\end{mdframed}

\begin{enumerate}
	\load
	\item On the basis on the temperature and pressure values obtained before, compute the
	      altitude in m thanks to Equation~\eqref{eq:altitude}. The result must then be published on a new
	      topic \mytext{/altitude} \textbf{with a rate of 10 Hz}.
	      \save
	      \begin{solution}\textbf{\textcolor{red}{4 points, on vérifie surtout l'agorithmie et la logique du code. 1.5 points si le publisher est correctement déclaré.}}
		      Du coup, ici il faut que les étudiants définissent un Publisher, avec une boucle de
		      publication qui doit tourner à 10Hz à l'aide de \texttt{rospy.Rate} et
		      \texttt{rospy.sleep}. La difficulté/piège ici est que les 2 topics de température et
		      pression ne publient pas à la même vitesse : il faut donc que les étudiants trouvent un
		      moyen pour stocker la solution dans une variable utilisée pour calculer l'altitude avant
		      la publication sur le topic. Dans le code \texttt{exam\_node.py} j'utilise 2 variables
		      globales pour faire ça, et ce sont ces variables qui sont utilisées dans la boucle de
		      publication. D'autres solutions doivent être possibles, en particulier en utilisant un
		      objet.
	      \end{solution}
\end{enumerate}
The following questions are mostly independant.
\begin{enumerate}
	\load
	\item Write a launchfile in the \mytext{exam} package launching both
	      \mytext{mydrone\_node} and \mytext{exam\_node} nodes.
	      \begin{solution}\textbf{\textcolor{red}{1 point. La moitié si pas de commentaires.}}
		      Rien de spécial ici :
		      \begin{verbatim}
<launch>

<!-- Run one mydrone nodes -->
<node pkg="mydrone" name="mydrone" type="mydrone_node.py"/>

<!-- Run one exam node -->
<node pkg="exam" name="exam" type="exam_node.py" output="screen" required="true">

</node>

</launch>			
		\end{verbatim}
	      \end{solution}
	\item On the basis on the altitude computed before, trigger the call to the
	      \mytext{/altitude\_warning} service to change the color of the \texttt{ALT WARN} indicator
	      to red or green.
	      \begin{solution}
		      \textbf{\textcolor{red}{4 points. 2 points si appel au service en boucle.}}
		      Rien de spécial ici, il faut appeler le service uniquement si on passe dessus ou dessous
		      l'altitude seuil. ATTENTION : il faut appeler le service SEULEMENT à la frontière, et ne
		      pas appeler non-stop le service !
	      \end{solution}
	\item The company plans to propose multiple model of the remote controller in the future, so
	      that the current altitude limit of 1300m could be extended.

	      \begin{enumerate}
		      \item To propose a more generic way to deal with these various models of remote, add a
		            \texttt{threshold} parameter in the launchfile, and use it in your code to make the
		            \texttt{ALT WARN} indicator turn red depending on this parameter value.
		            \begin{solution}\textbf{\textcolor{red}{1 points. 0 si la variable n'est pas initialisée.}}
			            Rien de spécial, il faut ajouter dans le launchfile :
			            \begin{verbatim}
<!-- Parameters -->
<param name="threshold" value="1300" />								  
						\end{verbatim}
			            Et ajouter le \texttt{getparam} dans le code. Attention, la variable doit
			            être initialisée !
		            \end{solution}
		      \item In practice, one would the altitude warning to be triggered only if the drone stays above
		            the \texttt{threshold} \textbf{more than 5s}. Modify your code to implement such a delay. The
		            \texttt{ALT WARN} indicator must still go back to green as soon as the drone flights above the
		            \texttt{threshold}.
		            \begin{solution}\textbf{\textcolor{red}{3 points.}}
			            Euh ... j'ai eu la flemme de coder ça, mais il suffit de mettre un compteur
			            dans la boucle de publication qui compte combien de fois on a été au dessus
			            de l'altitude seuil. Connaissant la vitesse de la boucle (10ms), il faut
			            donc avoir été 500 fois au dessus pour appeler le service. Bien sûr, le
			            compteur doit être réinitialisé dès qu'on repasse sous le seuil.
		            \end{solution}
	      \end{enumerate}


\end{enumerate}

\begin{solution}
	Il reste à évaluer la qualité du code :
	\begin{itemize}
		\item est-ce que le neud se lance et fait ce qu'il doit faire, selon l'avancée de l'étudiant ? \textbf{\textcolor{red}{2 point}}
		\item est-ce que le code a été commenté partout et les commentaires sont clairs ?
		      \textbf{\textcolor{red}{4 points. La moitié si assez peu de commentaires et même 0 s'il n'y
				      en a clairement pas assez.}}
		\item qualité du code : algorithmiquement, est-ce "bien" codé, en respectant les bonnes pratiques ? \textbf{\textcolor{red}{2 points.}}
	\end{itemize}

\end{solution}

\end{document}
