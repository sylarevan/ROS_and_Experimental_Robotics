\documentclass[conference]{IEEEtran}

\usepackage{cite}
\usepackage{amsmath,amssymb,amsfonts}
\usepackage{algorithmic}
\usepackage{graphicx}
\usepackage{textcomp}
\usepackage{xcolor}
\def\BibTeX{{\rm B\kern-.05em{\sc i\kern-.025em b}\kern-.08em
    T\kern-.1667em\lower.7ex\hbox{E}\kern-.125emX}}
\begin{document}

\title{My ROS project}

\author{
    \IEEEauthorblockN{Nicolas \textsc{Dupont}}
    \IEEEauthorblockA{Master ISI/SAR/SMR\\
        \textit{Student number 12345678}
    }
    \and
    \IEEEauthorblockN{Robert \textsc{Dupond}}
    \IEEEauthorblockA{Master ISI/SAR/SMR\\
        \textit{Student number 12345678}
    }
}

\maketitle

\begin{abstract}
    This is my wonderful \LaTeX~report on my ROS project. In this document, we will \ldots
\end{abstract}

\section{Introduction}
This doubled-sided, 2 columns document must not exceed 3 pages, and should be written in English. It
should be constituted of a limited number of figures and tables, and be oriented towards the
analysis of your architecture and of its performance. It is \emph{not} a purely technical report, we
expect you to explain in this document your choices, to analyze their consequences, mainly in terms
of performance w.r.t.\ the task to be solved.

The plan of the document (section/subsection organization) must not be changed. Neverthlesse, you
can eventually add some subsections if really required.

\section{Presentation of the ROS architecture}
In this section, you shall present the ROS architecture you build to solve the objectives you
mentioned in the introduction. This presentation must be technical and go in depth when needed, so
as to demonstrate you ability to master ROS concepts and to use them in an actual project.

\subsection{A short overview}
% -> présentation de l'architecture
% => incluant forcément un schéma complet, détaillé, légendé de la topologie du système
In this subsection, you can first briefly introduce your architecture: used nodes, topics, messages,
services, etc. Then, you shall precisely describe some elements, and include and comment a mandatory
figure/sketch like in Figure~\ref{fig:architecture} representing the architecture topolgy, i.e.\
your nodes, how they communicate with each other, etc.
%
\begin{figure}[htbp]
    \centerline{\includegraphics{fig1.png}}
    \caption{My very simple architecture.}
    \label{fig:architecture}
\end{figure}
%

\subsection{Algorithmic structure of the architecture}
% => incluant forcément des précisions sur l'agorithmie pour un scénario donné (tel nœud détecte un
%    obstacle, envoie un message, qui stoppe le robot, qui passe en attente, qui ...), exemple =
%    machine à état (si pas de lignes, alors ..., ou si lignes ET mur, alors ...)
In this subsection, you must detail how all the nodes listed in your architecture work together
for a given scenario. For instance:
\begin{itemize}
    \item node A detects an obstacle on the basis on node B which compute the mean distance to a
          laser impact in front of the robot from the LDS sensor ;
    \item then, node C decides to stop the robot movement, and sends a request to the service
          exposed by node D ;
    \item so, node D decides wether the robot should (choice 1) stay at rest or (choice 2) move away
          from the obstacle:
          \begin{itemize}
              \item if choice 1: node D sends \ldots
              \item if choice 2: node D requests \ldots
          \end{itemize}
    \item etc.
\end{itemize}
%
Such an algorithm, here described in the form of if/else/then statments, can also be described
through a state machine that you should carefully and precisely formalize. You can obviously use an
additional figure to support your discussions, if needed.

\section{Performance characterization}
% -> caractérisation des performances (avec définition et calcul des indicateurs, incluant si
%    nécessaire des figures supplémentaires, en nombre très restreint)
In this section, you must analyzed the performance of your architecture. Such an analysis requires
first the formal definition of indicators/cues (e.g. mean errors, response time, etc., like the
fictitious $H^{\infty}$ definition in Equation~\eqref{eq:def_h_infty}) of your choice, which must be correctly chosen so as to
assess the performance you are trying to evaluate (e.g.\ precision of the line following algorithm,
stability of the servoing w.r.t.\ very curved lines, etc.).
%
\begin{equation}
    H^{\infty} = \alpha + \Gamma.
    \label{eq:def_h_infty}
\end{equation}
%
Once chosen and defined, you have then to plot --in a specific figure or on a table
like in Table~\ref{tab:performance_table}-- the values of your indicator and comment its evolution
and signification regarding the performance you are trying to reach.
%
\begin{table}[htbp]
    \caption{My super performance indicator}
    \begin{center}
        \begin{tabular}{|c|c|c|c|}
            \hline
            \textbf{Table} & \multicolumn{3}{|c|}{\textbf{Table Column Head}}                                                         \\
            \cline{2-4}
            \textbf{Head}  & \textbf{\textit{Table column subhead}}           & \textbf{\textit{Subhead}} & \textbf{\textit{Subhead}} \\
            \hline
            $H^{\infty}$   & 42\%$^{\mathrm{a}}$                              & 67\%                      & 89\%                      \\
            \hline
            \multicolumn{4}{l}{$^{\mathrm{a}}$Only when it works.}
        \end{tabular}
        \label{tab:performance_table}
    \end{center}
\end{table}


Feel free here to add subsection if it helps in understanding the different steps in your
evaluation of your architecture performance.

\section{Conclusion and perspectives}
% -> limites et perspectives (qu'est ce qui peut être amélioré, comment, pourquoi, etc.)
You can use this last section to conclude on your work and explain what you could have improved if
you had more time, how and why.

\end{document}
